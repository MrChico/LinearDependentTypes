\documentclass[a4paper,english]{lipics-v2018}
%\usepackage{tikz-cd}
%\usepackage[inference]{semantic}
%8\usepackage{float}
%This is a template for producing LIPIcs articles. 
%See lipics-manual.pdf for further information.
%for A4 paper format use option "a4paper", for US-letter use option "letterpaper"
%for british hyphenation rules use option "UKenglish", for american hyphenation rules use option "USenglish"
% for section-numbered lemmas etc., use "numberwithinsect"

\usepackage{microtype}%if unwanted, comment out or use option "draft"
\bibliographystyle{plainurl}% the recommnded bibstyle
\title{The diagram model of linear dependent type theory}

\titlerunning{Models of linear dependent type theory}%optional, please use if title is longer than one line

\author{Martin Lundfall}{Stockholm University}{martin@dapp.org}{}{}%{johnqpublic@dummyuni.org}{https://orcid.org/0000-0002-1825-0097}{[funding]}%mandatory, please use full name; only 1 author per \author macro; first two parameters are mandatory, other parameters can be empty.

\authorrunning{M. Lundfall}%mandatory. First: Use abbreviated first/middle names. Second (only in severe cases): Use first author plus 'et. al.'

\Copyright{Martin Lundfall}%mandatory, please use full first names. LIPIcs license is "CC-BY";  http://creativecommons.org/licenses/by/3.0/
\subjclass{
\ccsdesc[500]{Theory of computation~Linear logic};
\ccsdesc[500]{Theory of computation~Type theory}
}
\keywords{Dependent type theory, linear type theory, diagram model, monoidal categories, groupoid model}%mandatory

\category{}%optional, e.g. invited paper

\relatedversion{\url{http://kurser.math.su.se/pluginfile.php/16103/mod_folder/content/0/2017/2017_47_report.pdf}}%optional, e.g. full version hosted on arXiv, HAL, or other respository/website

\supplement{}%optional, e.g. related research data, source code, ... hosted on a repository like zenodo, figshare, GitHub, ...

\funding{}%optional, to capture a funding statement, which applies to all authors. Please enter author specific funding statements as fifth argument of the \author macro.

\acknowledgements{I want to thank Peter Lumsdaine for his constant guidance and inspiration.}%optional

%Editor-only macros:: begin (do not touch as author)%%%%%%%%%%%%%%%%%%%%%%%%%%%%%%%%%%
\EventEditors{John Q. Open and Joan R. Access}
\EventNoEds{2}
\EventLongTitle{42nd Conference on Very Important Topics (CVIT 2016)}
\EventShortTitle{CVIT 2016}
\EventAcronym{CVIT}
\EventYear{2016}
\EventDate{December 24--27, 2016}
\EventLocation{Little Whinging, United Kingdom}
\EventLogo{}
\SeriesVolume{42}
\ArticleNo{23}
\nolinenumbers %uncomment to disable line numbering
%\hideLIPIcs  %uncomment to remove references to LIPIcs series (logo, DOI, ...), e.g. when preparing a pre-final version to be uploaded to arXiv or another public repository
%%%%%%%%%%%%%%%%%%%%%%%%%%%%%%%%%%%%%%%%%%%%%%%%%%%%%%

\begin{document}

\maketitle

\begin{abstract}
We present a type theory dealing with non-linear, ``ordinary'' dependent types (which we will call \textit{cartesian}), and \textit{linear types}, where both constructs may depend on terms of the former. In the interplay between these, we find the new type formers $\sqcap_{x : A}B$ and $\sqsubset_{x : A}B$, akin to $\Pi$ and $\Sigma$, but where the dependent type $B$, (and therefore the resulting construct) is a linear type. These can be seen as internalizing universal and existential quantification over linear propositions, respectively. We also consider two modalities, $M$ and $L$, transforming linear types into cartesian types and vice versa.

The theory is interpreted in a split comprehension category $\pi : \mathcal{T} \to \mathcal{C}^\to$ \cite{jacobs}, accompanied by a split monoidal fibration, $q : \mathcal{L} \to \mathcal{C}$. We interpret $\mathcal{C}$ as a category of contexts, which for any $\Gamma \in \mathcal{C}$, determines the fibers $\mathcal{T}_\Gamma$ and $\mathcal{L}_\Gamma$. We interpret $\mathcal{T}_\Gamma$ as category of the cartesian types over $\Gamma$, and $\mathcal{L}_\Gamma$ as the monoidal category of linear types in $\Gamma$. In this setting, the type formers $\sqcap_{x :A}$ and $\sqsubset_{x : A}$ are understood as right and left adjoints of the monoidal reindexing functor $\pi_A^* : \mathcal{L}_\Gamma \to \mathcal{L}_{\Gamma.A}$ corresponding to the weakening projection $\pi_A : \Gamma.A \to \Gamma$ in $\mathcal{C}$. The operators $M$ and $L$ give rise to a fiberwise adjunction $L \dashv M$ between $\mathcal{L}$ and $\mathcal{T}$, where we understand the traditional exponential modality as the comonad $! = LM$.

We provide a model of this theory called the \textit{Diagram model}, which extends the groupoid model of dependent type theory \cite{hofmann1998} to accomodate linear types. Here, cartesian types over a context $\Gamma$ are interpreted as a family of groupoids indexed over the groupoid $\Gamma$, while linear types are interpreted as diagrams over groupoids, $A : \Gamma \to \mathcal{V}$ in any symmetric monoidal category $\mathcal{V}$. We show that the \textit{diagrams model} can under certain conditions support a linear analogue of the univalence axiom, and provide some discussion on the higher-dimensional nature of linear dependent types.
\end{abstract}

\bibliography{LundfallCSL}


\end{document}
